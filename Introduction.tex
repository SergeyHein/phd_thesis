Based on the work by Wriggers and Miehe \cite{WrMi94}, Peric and Owen \cite{PeOw98}, and Costabel and Stephan \cite{CoSt88,CoSt90} we introduce finite element (FE), boundary element (BE) and FE/BE coupling procedures for friction contact problems in elastoplasticity. In our approach we use the radial return algorithm (see Simo and Hughes \cite{SiHu98}, Simo and Miehe \cite{SiMi92}) for both plastification of the material and contact. Here, we study small deformations and therefore can model the linear elastic parts by standard BEM with the linear elastic fundamental solution. Our numerical results demonstrate clearly that   pure FEM,  pure BEM and FE/BE coupling approaches give relevant numerical simulations. 

 The framework of Glowinski \cite{Glo2000} supplies an abstract and a numerical (FE)  analysis  for nonlinear variational problems.  The work of Eck and Jaru\v{s}ek \cite{EckJa98} provides existence and regularity results for the static one body contact problem with Coulomb friction. The existence, uniqueness and regularity results for boundary value problems of the plastic flow theory are given in the book by Korneev and Langer \cite{KoLa84}. This work also provides foundations for FE analysis of quasistatic plastic flows.  Existence, uniqueness and stability results are obtained in the work of Han and Reddy \cite{WeRe99} for the one body associated elastoplastic problem. Moreover, they prove the convergence results for discrete  versions. The work of Blaheta et. al.  \cite{Bl97,BlAx97,AxBlKo97} is devoted to the investigation of convergence of discretized problems, namely convergence of the Newton and Newton-like methods for FE discretization of the one body associated elastoplasticity problem. 

 For the theoretical background of the boundary integral equations and  the Galerkin boundary element methods (BEM) for linear problems we refer to the book of Sauter and Schwab \cite{SaSch04}. The coupling technique of boundary element method and finite element methods are described in work by Stephan \cite{Ste04}, Carstensen and Stephan \cite{CaSt95}. In the works of Brebbia et. al.  \cite{Br80,BrWa80,Br81,Br83} one can find extension of the boundary element techniques for solving nonlinear elastoplastic problems. These approaches are based on  the heuristical collocation method. An application of the boundary element method to  elastoplastic unilateral contact problems with friction was suggested by  Polizzotto and Zitto \cite{PoZi98}. Theoretical and numerical investigations for the one-body quasistatic elastoplastic problem are done by Alberty \cite{AlbertyPhD}. Theoretical and numerical investigations of the time-discretized one-body quasistatic elastoplastic  problem with a non-penetration contact constraint are done by Zarrabi \cite{ZarrabiPhD}.

Based on the series of papers by Mukherjee \cite{Mu82}, Chandra and  Mukherjee \cite{MuChaDBE3,MuCha84,MuCha84a} we introduce  finite element, boundary element and FE/BE coupling procedures for metal-forming and metal chipping.  As in \cite{Ha76,Ha82,LeMa77} we consider Hart's constitutive model, which describes  hypoelasto-viscoplasticity \cite{Belytschko2000}. We use the updated Lagrange approach in order to pose the equilibrium equation of the media, i.e. the equilibrium equation of the body and constitutive conditions on the time interval $(t,t+dt)$ are given employing the pure Lagrange approach with the reference configuration coinciding with the actual one taken at time $t$. Discretizing the problem in time one obtains the set of problems at discrete time points $t_n$, whereas the mesh has to be updated corresponding to the updated Lagrange description as soon as the new actual configuration is known.



This thesis is organized as follows. In \textbf{Chapter \ref{chap:SmallDeformations}} we consider two-body contact problems in elastoplasticity with and without friction and present solution procedures based on finite element and boundary element methods.  We formulate the weak elastoplastic contact problem in Section \ref{sec:ElPlContact:WeakPenalty} and derive its penalty approximation. We discretize  the penalty weak formulation in time as well as in space in Section \ref{sec:ElPlContact:DiscretizationSolutionProcedure}. \\ The predictor-corrector solution procedure for the elastoplastic contact problem is considered in Section \ref{sec:ElPlContact:WeakPenalty}.   The radial return mapping algorithm is used to handle both contact conditions and plastification. We describe in detail a segment-to-segment contact discretization, which allows also to model friction. \\ The linearization of contact and plastic terms in the equilibrium equation is derived in Section \ref{sec:ConstitutiveConditions:Discretization}. In Section  \ref{sec:ElPlContact:DiscretizationSolutionProcedure} we provide the FEM/FEM, BEM/BEM and FEM/BEM, respectively,  discretization procedures of two-body elastoplastic frictional contact (Problem \ref{prob:ElPlWeakRegularizedContactTimeDiscretization}). In Section \ref{sec:ContactFunctionalInvestigation} we extend the return mapping algorithm for elastoplasticity, which  is carried out in \cite{Bl97} in order to  investigate the contact return mapping algorithm.  In Section \ref{sec:NewtonTypeMethod} we prove the convergence of the Newton method introduced in Section \ref{sec:ElPlContact:DiscretizationSolutionProcedure} for elastoplasticity with frictional contact using the results obtained in Section \ref{sec:ContactFunctionalInvestigation} and in \cite{Bl97}. In Section \ref{sec:NewtonLikeIteration} we extend  the Newton-like iterations introduced in \cite{BlAx97}  onto elastoplasticity with frictional contact.  Using the results obtained in Section \ref{sec:ContactFunctionalInvestigation} and in \cite{BlAx97} we prove the convergence of extended Newton-like iterations.  The approaches given cover small deformations. Numerical simulations in Section \ref{sec:FF:BB:FB:Benchmark}  demonstrate the wide applicability of our approaches described in Sections \ref{sec:FEMFEM}, \ref{sec:BEMBEM}, \ref{sec:FEMBEM}. In Section \ref{sec:ElPlContact:DomainDecomposition} we extend the coupling procedures introduced in Section \ref{sec:ElPlContact:DiscretizationSolutionProcedure}. We decompose one body (that is subjected to the elastoplastic material law) into a  purely elastic domain and an elastoplastic domain. Using Lagrange multipliers (cf. \cite{W00}) on the interface boundary we obtain a coupling formulation. Section \ref{sec:DomainDecomposition:Benchmarks} is devoted to the FE-BE-FE (elastic body is discretized with BE; elastoplastic body is decomposed into 2 subdomains, the linear elastic is with  FE, whereas the elastoplastic with BE) simulations of our approaches given in Section \ref{sec:ElPlContact:DomainDecomposition:WeakFormulation}. In Section \ref{sec:ThElContact} we consider a two-body thermo-elastic frictional contact problem, as in Sections \ref{sec:ElPlContact:DiscretizationSolutionProcedure}, \ref{sec:ElPlContact:DomainDecomposition} the contact conditions are regularized using the penalty method. We end up Section \ref{sec:ThElContact} with Subsection \ref{sec:ThElContact:Benchmark}, where we present the numerical simulation of the solution procedure introduced in Section \ref{sec:decomp}.


In \textbf{Chapter \ref{chap:HypoelastoViscoplasticity}} we consider the application of Hart's model for hypoelasto-viscoplasticity to  contact problems. We start by providing a theoretical background of a continuum mechanic description of large deformations (Section \ref{sec:LargeViscoplacticity:EquilibriumEquation}) as well as Hart's constitutive equations (Section \ref{sec:HartConstitutiveLaw}). The integration of the material law is done via an explicit finite difference scheme in time in Section \ref{sec:LargeViscoplacticity:HartModelTimeIntegration}. We apply the updated Lagrange approach in Section \ref{sec:LargeViscoplacticity:UpdatedLagrangeApproach}. Employing the update Lagrange approach we derive in Section \ref{sec:BEM_HyperElasto_VP}  the space discretization using BE discretization of the problem posed in Section \ref{sec:LargeViscoplacticity:EquilibriumEquation} under Hart's material law, which  is integrated in time in Section \ref{sec:LargeViscoplacticity:HartModelTimeIntegration}. In Section \ref{sec:BEM:Benchmarks} we present a  numerical simulation: stretching a square plate. The problem is discretized using FE and BE methods. In Section \ref{sec:FEMBEM_HyperElasto_VP} we consider a two-body contact problem under Hart's constitutive conditions coupled with heat conduction. In Section \ref{sec:FEMBEM_HyperElasto_VP} we pose the thermo-mechanical weak formulation in rate form. In Section \ref{sec:Benchmarks_HyperElasto_VP} we present two benchmark problems: in Example 1 we consider  FEM-BEM coupling for one a body problem; in Example 2 we consider the metal chipping process using FEM-BEM coupling, whereas the work tool is discretized with BE and work piece with FE in space. The work tool in our simulation is supposed to be purely elastic. 

The appendix is devoted to the implementation of the boundary integral operators, the volume potentials as well as the finite element method for large deformations. 
