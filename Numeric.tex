\section{Implementation}
In this section we present some tools necessary for the implementation of  numerical methods presented in this thesis. Implementation techniques of Boundary and Volume Integral Operators needed for BEM for elastoplastic 2-body contact Section \ref{sec:BEMBEM} discussed in Section \ref{sec:Numeric:BEM:Operators}. In Section \ref{sec:Numeric:LargeStrain} we explain the computation of stiffness matrixes for hypoelasto-viscoplasticity under Hart's model using software package {\tt maiprogs}. The implementation is done by the author of this thesis as an  internal library   of software package {\tt maiprogs} \cite{Mai99prc,Mai96prd,Mai01a}. If you have an original version of the package you can find the latest documentation in the subdirectory \texttt{doku}.
\subsection{Boundary operators and volume potentials}\label{sec:Numeric:BEM:Operators}
We consider an elastoplatic body occupying a Lipschitz domain $\Omega\subset\R^{d}$, $d=2,3$ with boundary $\Gamma:=\partial \Omega$. If the body is in equilibrium, then for all $x\in\Omega$  the displacement vector and the stress tensor at $x$ are uniquely determined by displacements and tractions on the boundary $\Gamma$, body forces and plastic strains in the domain $\Omega$. These relations are given by mean of boundary integral operators and Newton volume potentials as follows (for convenience we present componentwise representation).
\begin{eqnarray}
u_i(x)&=&\int_{\Gamma}u^{*}_{ij}(x,y) p_j(y)-\int_{\Gamma}p^{*}_{ij}(x,y) u_j(y)+\int_{\Omega}u^{*}_{ij}(x,y)b_{j}(y)\nonumber \\ &+&\int_{\Omega}\hat{\sigma}^{*}_{jki}(x,y)\varepsilon^{a}_{jk}(y) \label{eq:RepresentationFormulaDisplacement}\\
\sigma_{ij}(x)&=&\int_{\Gamma}u^{*}_{ijk}(x,y) p_k(y)-\int_{\Gamma}p^{*}_{ijk}(x,y) u_k(y)+\int_{\Omega}u^{*}_{ijk}(x,y)b_{k}(y)\nonumber \\
&+&\int_{\Omega}\hat{\sigma}^{*}_{ijkl}(x,y)\varepsilon^{a}_{kl}(y)+f_{ij}(\varepsilon^{a}_{kl}) \label{eq:RepresentationFormulaStress}\\
f_{ij}(\varepsilon^{a}_{ij})&:=&-\frac{G}{4(1-\nu)}[2\varepsilon^{a}_{ij}+(1-4\nu)e\delta_{ij}]
\end{eqnarray}
Here and later on the summation is applied with respect to repeated index. Indexes $i$ , $j$, $k$, $l$ runs from $1$ to $3$ in 3D and take values $1$ and $2$ in 2D. The kernels $p^{*}_{ij}$, $u^{*}_{ij}$, $\hat{\sigma}^{*}_{jki}$, $u^{*}_{ijk}$,  $\hat{\sigma}^{*}_{ijkl}$ are defined in next section. $e:=\varepsilon^{a}_{11}+\varepsilon^{a}_{22}+\varepsilon^{a}_{33}$ in 3D case and plain strain case in 2D.
\subsubsection{Definition of kernels}

\begin{definition} Kernel $u^{*}_{ij}$ corresponds to the Single Layer Potential $(V\psi)_{i}(x)=\Int_{\Gamma}u^{*}_{ij}(x,y)\psi_{j}(y)\,d\Gamma_{y}$, where $\psi\in \Hb^{-1/2}(\Gamma)$, $x\in\Gamma\subset \R^{2}$ and to the Newton potential $(N_{0}\psi)_{i}(x)=\Int_{\Omega}u^{*}_{ij}(x,y)\psi_{j}(y)\,d\Omega_{y}$, where $\psi\in \Hb^{1}(\Omega)$, $x\in\Gamma\subset \R^{2}$ 
\begin{equation}\nonumber
u^{*}_{ij}(x,y):=\frac{1}{16 \pi (1-\nu)G |x-y|}\left\{ (3-4\nu)\delta_{ij}+\frac{(x_i-y_i)(x_j-y_j)}{|x-y|^2}\right\} \quad 3D,
\end{equation}
\begin{equation}\nonumber
u^{*}_{ij}(x,y):=\frac{-1}{8 \pi (1-\nu)G }\left\{ (3-4\nu)\mbox{ln}|x-y|\delta_{ij}-\frac{(x_i-y_i)(x_j-y_j)}{|x-y|^2}\right\} \quad 2D.
\end{equation}
\end{definition}

\begin{definition}Kernel  $p^{*}_{ij}$ corresponds to the Double Layer Potential $(K\psi)_{i}(x)=\Int_{\Gamma}p^{*}_{ij}(x,y)\psi_{j}(y)\,d\Gamma_{y}$, where $\psi\in \Hb^{1/2}(\Gamma)$, $x\in\Gamma\subset \R^{2}$.
\begin{eqnarray}
p^{*}_{ij}(x,y):=\frac{1}{4\alpha\pi(1-\nu)|x-y|^{\alpha}}&&\left\{\left[(1-2\nu)\delta_{ij}+\beta\frac{(x_i-y_i)(x_j-y_j)}{|x-y|^2}\right]\frac{(x_k-y_k)n_k}{|x-y|}\right. \nonumber \\
&-&\left.(1-2\nu)\frac{(x_i-y_i)n_j-(x_j-y_j)n_i}{|x-y|}\right\}. \nonumber
\end{eqnarray}
\end{definition}
Recall that $p^{*}_{ij}(x,y)=\left(T_y G(x,y)\right)^T$ with the fundamential solution $G$ of the Lame operator. Where $\alpha=2,1$, $\beta=3,2$ for three- two-dimensional plane strain, respectively. 


\begin{definition}\label{def:sigma*:jkl} Kernel $\sigma^{*}_{jki}$\\
\begin{eqnarray}
\sigma^{*}_{jki}(x,y):=\frac{1}{4\alpha \pi (1-\nu) |x-y|^{\alpha}} \{ (1-2\nu)(\frac{x_k-y_k}{|x-y|}\delta_{ij}&+&\frac{x_j-y_j}{|x-y|}\delta_{ki}-\frac{x_i-y_i}{|x-y|}\delta_{jk}) \nonumber \\
 &+&\beta\frac{(x_i-y_i)(x_j-y_j)(x_k-y_k)}{|x-y|^3}\}, \nonumber
\end{eqnarray}
where $\alpha=2,1$, $\beta=3,2$ for three- two-dimensional plane strain, respectively.
\end{definition}
Hence, Definition \ref{def:sigma*:jkl} in 2D gives
\begin{eqnarray}
\sigma^{*}_{ijk}(x,y)&=&\frac{1}{4 \pi (1-\nu) } \{ (1-2\nu)(-\frac{\partial}{\partial y_j}\log|x-y|\delta_{ki}-\frac{\partial}{\partial y_i}\log|x-y|\delta_{jk}+\frac{\partial}{\partial y_k}\log|x-y|\delta_{ij}) \nonumber \\
&+&
\frac{\partial}{\partial y_k}\left[\frac{(x_i-y_i)(x_j-y_j)}{|x-y|^2}\right]-\delta_{ik}\frac{\partial}{\partial y_j}\log|x-y|-\delta_{jk}\frac{\partial}{\partial y_i}\log|x-y| \}\nonumber \\
&=&\frac{1}{4 \pi (1-\nu) } \{- 2(1-\nu)(\frac{\partial}{\partial y_j}\log|x-y|\delta_{ki}+\frac{\partial}{\partial y_i}\log|x-y|\delta_{jk}) \nonumber \\
&+&(1-2\nu)\frac{\partial}{\partial y_k}\log|x-y|\delta_{ij})+\frac{\partial}{\partial y_k}\left[\frac{(x_i-y_i)(x_j-y_j)}{|x-y|^2}\right] \} \mbox{ 2D}.\nonumber
\end{eqnarray}

\begin{remark}
In 3D $\hat{\sigma}^{*}_{jki}(x,y)=\sigma^{*}_{jki}(x,y)$. But not in 2D plain strain configuration. If the trace of nonelastic part of strain equals zero then: 
\begin{equation}
\hat{\sigma}^{*}_{jki}(x,y)=\sigma^{*}_{jki}(x,y)-\frac{2\nu (x_i-y_i)}{4\pi (1-\nu ) |x-y|^2}\delta_{jk}.
\end{equation}
or
\begin{equation}
\hat{\sigma}^{*}_{ijk}(x,y)=\sigma^{*}_{ijk}(x,y)-\frac{2\nu (x_k-y_k)}{4\pi (1-\nu ) |x-y|^2}\delta_{ij}.
\end{equation}
then
\begin{eqnarray}
\hat{\sigma}^{*}_{ijk}(x,y)=\frac{1}{4 \pi (1-\nu) |x-y|} \{ (1-2\nu)(\frac{x_j-y_j}{|x-y|}\delta_{ki}&+&\frac{x_i-y_i}{|x-y|}\delta_{jk})-\frac{x_k-y_k}{|x-y|}\delta_{ij} \nonumber \\
 &+&2\frac{(x_k-y_k)(x_i-y_i)(x_j-y_j)}{|x-y|^3}\} \nonumber
\end{eqnarray}
or
\begin{eqnarray}
\sigma^{*}_{ijk}(x,y)&=&\frac{1}{4\alpha \pi (1-\nu) } \{- 2(1-\nu)(\frac{\partial}{\partial y_j}\log|x-y|\delta_{ki}+\frac{\partial}{\partial y_i}\log|x-y|\delta_{jk}) \nonumber \\
&+&\frac{\partial}{\partial y_k}\log|x-y|\delta_{ij}+
\frac{\partial}{\partial y_k}\left[\frac{(x_i-y_i)(x_j-y_j)}{|x-y|^2}\right] \} \mbox{ 2D}.\nonumber 
\end{eqnarray}


For pure thermal strains one has
\begin{equation}
\hat{\sigma}^{*}_{jki}(x,y)=\sigma^{*}_{jki}(x,y)+\frac{\nu (x_i-y_i)}{4\pi (1-\nu ) |x-y|^2}\delta_{jk}.
\end{equation}
\end{remark}

\begin{definition} Kernel $u^{*}_{ijk}(x,y):=-\sigma^{*}_{ijk}$ corresponds to the Newton potential \linebreak $(N_{1}\psi)_{ij}(x)=\Int_{\Omega}u^{*}_{ijk}(x,y)\psi_{k}(y)\,d\Omega_{y}$, where $\psi\in \Hb^{1}(\Omega)$, $x\in\Gamma\subset \R^{2}$. 
\end{definition}

\begin{definition}\label{eq:KernelPijk} Kernel $p^{*}_{ijk}$\\
\begin{eqnarray}
p^{*}_{ijk}(x,y)&:=&\frac{G}{2\alpha \pi (1-\nu) |x-y|^{\beta}}\{\beta \frac{(x_l-y_l)n_l}{|x-y|}[(1-2\nu)\delta_{ij}\frac{x_k-y_k}{|x-y|} \nonumber \\[2ex]
&+&\nu(\delta_{ik}\frac{x_j-y_j}{|x-y|}+\delta_{jk}\frac{x_i-y_i}{|x-y|})- \gamma \frac{(x_i-y_i)(x_j-y_j)(x_k-y_k)}{|x-y|^3}] \nonumber \\[2ex]
&+&\beta\nu(n_i \frac{(x_j-y_j)(x_k-y_k)}{|x-y|^2}+n_j\frac{(x_i-y_i)(x_k-y_k)}{|x-y|^2}) \nonumber \\[2ex]
&+&(1-2\nu)(\beta n_k \frac{(x_i-y_i)(x_j-y_j)}{|x-y|^2}+n_j \delta_{ik} +n_i \delta_{jk}) - (1-4\nu)n_k \delta_{ij}\},
\end{eqnarray}
where $\alpha=2,1$, $\beta=3,2$, $\gamma=5,4$ for three and two dimensions respectively. 
\end{definition}

\begin{definition} Kernel $\sigma^{*}_{ijkl}$\\
\begin{eqnarray}
\sigma^{*}_{ijkl}&:=&\frac{G}{2\alpha \pi (1-\nu)r^{\beta}}\{\beta(1-2\nu)(\delta_{ij}r_{,k}r_{,l}+\delta_{kl}r_{,i}r_{,j}) \nonumber \\
&+& \beta\nu(\delta_{li}r_{,j}r_{,k}+\delta_{jk}r_{,l}r_{,i}+\delta_{ik}r_{,l}r_{,j}+\delta_{jl}r_{,i}r_{,k})-\beta\gamma r_{,i}r_{,j}r_{,k}r_{,l} \nonumber \\
&+& (1-2\nu)(\delta_{ik}\delta_{lj}+\delta_{jk}\delta_{li})-(1-4\nu)\delta_{ij}\delta_{kl} \}, \nonumber
\end{eqnarray}
where $\alpha=2,1$, $\beta=3,2$, $\gamma=5,4$ for three-dimensions and plane strain, respectively, and
\begin{equation}\nonumber
r_{,i}:=\frac{\partial |x-y|}{\partial y_i}=\frac{y_i-x_i}{|x-y|}.
\end{equation}
In case $ \mbox{ trace } \varepsilon^a=0$
\begin{equation}
\hat{\sigma}^{*}_{ijkl}(x,y)=\sigma^{*}_{ijkl}(x,y)+\frac{G}{2\pi(1-\nu)r^2}[4\nu r_{,i}r_{,j}\delta_{kj}-2\nu \delta_{ij}\delta_{kl}],
\end{equation}
\begin{equation}
f_{ij}=-\frac{G}{4(1-\nu)}[2 \varepsilon^{a}_{ij}+(1-4\nu) \varepsilon^{a}_{ll}\delta_{ij}].
\end{equation}
For pure thermal strains one has
\begin{equation}
\hat{\sigma}^{*}_{ijkl}(x,y)=\sigma^{*}_{ijkl}(x,y)-\frac{G}{2\pi(1-\nu)r^2}[2\nu r_{,i}r_{,j}\delta_{kj}-\nu \delta_{ij}\delta_{kl}]
\end{equation}
\begin{equation}
f_{ij}=-\frac{G(1+\nu)}{1-\nu}\alpha T \delta_{ij} \quad \alpha -\mbox{ thermal coefficient, don't mix with another } \alpha
\end{equation} 
\end{definition}

In the next two subsections we will provide  regularization procedures for a boundary integral operator with the kernel $p^{*}_{ijk}$ and for a volume integral operator with the kernel $\sigma^{*}_{jki}$. The aim of the regularization is to reduce the order of a singularity of the strongly singular kernels, in order to simplify an implementation procedure.
\subsubsection{Regularization of $p^{*}_{ijk}$}
We regularize a boundary integral operator in (\ref{eq:RepresentationFormulaStress}) with a hyper-singular kernel $p^{*}_{ijk}$ (Definition
\ref{eq:KernelPijk}) employing an integration by part as follows
\begin{equation}\nonumber
p^{0*}_{ik}(x,y):= \frac{1}{2\pi}\frac{G}{1+\nu}\left[-\log|x-y|\delta_{ik}+\frac{(x_i-y_i)(x_j-y_j)}{|x-y|^2}\right]\mbox{ see \cite{Steinbach03} p. 157},
\end{equation}
we have
\begin{eqnarray}
\int_{\Gamma}p^{*}_{i0k}(x,y)u_k(y)& =& \frac{\partial}{\partial x_1} \int_{\Gamma}\frac{\partial}{\partial s_y}p^{0*}_{ik}(x,y) u_k(y), \nonumber \\
\int_{\Gamma}p^{*}_{i1k}(x,y)u_k(y)& =& -\frac{\partial}{\partial x_0} \int_{\Gamma}\frac{\partial}{\partial s_y}p^{0*}_{ik}(x,y)u_k(y), \nonumber 
\end{eqnarray}
or
\begin{eqnarray}
\int_{\Gamma}p^{*}_{i0k}(x,y)u_k(y)& =& - \int_{\Gamma}\left[\frac{\partial}{\partial y_1}\frac{\partial}{\partial s_y}p^{0*}_{ik}(x,y),\right]u_k(y) \nonumber \\
\int_{\Gamma}p^{*}_{i1k}(x,y)u_k(y)& =&  \int_{\Gamma}\left[\frac{\partial}{\partial y_0}\frac{\partial}{\partial s_y}p^{0*}_{ik}(x,y).\right]u_k(y) \nonumber 
\end{eqnarray}


\begin{remark}
In pure linear elasticity  there is no big difference in 2D and 3D. 
\end{remark}

\subsubsection{Regularization of volume integrals}
The volume integral in (\ref{eq:RepresentationFormulaDisplacement}) with a singular kernel $\sigma^{*}_{jki}$ admits straightforward implementation. The advantage of a regularization procedure, is that one can rewrite The volume integral an equivalent form as a sum of a boundary integral operator with a weakly-singular  kernel $u^{*}_{ij}$ and a volume integral operator with the same kernel.
\[
\sigma^{*}_{jki}(x,y)=G\left(u^{*}_{ij,k}+u^{*}_{ik,j}\right)+\frac{2G\nu}{1-2\nu}u^{*}_{il,l}\delta_{jk},
\]
where $u^{*}_{il,l}:=u^{*}_{i1,1}+u^{*}_{i2,2}$ in 2D, $u^{*}_{il,l}:=u^{*}_{i1,1}+u^{*}_{i2,2}+u^{*}_{i3,3}$ in 3D.

Hence,

\begin{equation}\nonumber
\int_{\Omega}\sigma^{*}_{jki} \varepsilon^a_{jk} d \Omega=\int_{\Omega}\left\{G\left(u^{*}_{ij,k}+u^{*}_{ik,j}\right)+\frac{2G\nu}{1-2\nu}u^{*}_{il,l}\delta_{jk}\right\} \varepsilon^a_{jk} d \Omega
\end{equation}
and after integrating by parts 
\begin{equation}\nonumber
\int_{\Omega}\sigma^{*}_{jki} \varepsilon^a_{jk} d \Omega=\int_{\Gamma}u^{*}_{ij}2G\left( \varepsilon^a_{jk}n_k+\frac{\nu}{1-2\nu} \varepsilon^a_{ll}\right)-\int_{\Omega}u^{*}_{ij}2G\left(\varepsilon^a_{jk,k}+\frac{\nu}{1-2\nu} \varepsilon^a_{ll,l}\right),
\end{equation}

\begin{equation}
u_i(x)=\int_{\Gamma}u^{*}_{ij}(x,y) \fictional{p}_j(y)-\int_{\Gamma}p^{*}_{ij}(x,y) u_j(y)+\int_{\Omega}u^{*}_{ij}(x,y)\fictional{b}_{j}(y),
\end{equation}
where
\begin{eqnarray}
\fictional{b}_j&=& b_j-2G\left( \varepsilon_{ij,i}^a+\frac{\nu}{1-2\nu}e_{,j}\right)= b_j- \sigma_{ij,i}^a \nonumber \\
\fictional{p}_i&=&b_i+2G\left(\varepsilon_{ij}^a n_j+\frac{\nu}{1-2\nu}e n_i\right)= \nonumber  p_i+\sigma_{ij}^a n_j
\end{eqnarray}


For plane problems (2D) these equations can also be used (i,j,k,l=1,2) with $e=\varepsilon_{11}+\varepsilon_{22}+\varepsilon_{33}$ in plane strain and $\nu$ replaced by $\bar{\nu}=\frac{\nu}{1+\nu}$ with $e=\varepsilon_{11}+\varepsilon_{22}$ in plane stress.

\begin{equation}\nonumber
\int_{\Omega}\hat{\sigma}^{*}_{jki} \varepsilon^a_{jk} d \Omega=\int_{\Omega}\sigma^{*}_{jki} \varepsilon^a_{jk}+\int_{\Omega}\frac{2\nu \delta_{jk}r_{,i}}{4\pi (1-\nu)r} \varepsilon^a_{jk} \quad \mbox{ if } e=0
\end{equation}
and after integrating by parts 
\begin{equation}\nonumber
\int_{\Omega}\sigma^{*}_{jki} \varepsilon^a_{jk} d \Omega=\int_{\Gamma}u^{*}_{ij}2G\left( \varepsilon^a_{jk}n_k+\frac{\nu}{1-2\nu} \varepsilon^a_{ll}\right)-\int_{\Omega}u^{*}_{ij}2G\left(\varepsilon^a_{jk,k}+\frac{\nu}{1-2\nu} \varepsilon^a_{ll,l}\right),
\end{equation}

\begin{eqnarray}
\sigma_{ij}(x)=\int_{\Gamma}u^{*}_{ijk}(x,y) \fictional{p}_k(y)-\int_{\Gamma}p^{*}_{ijk}(x,y) u_k(y)+\int_{\Omega}u^{*}_{ijk}(x,y)\fictional{b}_{k}(y)-C_{ijkl}\varepsilon^{a}_{kl}.\nonumber
\end{eqnarray}

\subsubsection{Boundary Elements - Plasticity}\label{sec:BEMPLAST}
' Poincar\'e-Steklov '
\begin{equation}
\begin{array}{c}
\left( \begin{array}{l}
 u \\
 Tu
 \end{array}
 \right)^{int}=\left( \begin{array}{cc}
 -K+\frac{1}{2} & V \\
 W & K'+\frac{1}{2}
 \end{array}
 \right)
 \left( \begin{array}{l}
  u \\
  \tilde{Tu}
 \end{array}
 \right)^{int}
 +\left( \begin{array}{l}
  \widetilde{N_0}f \\
  \widetilde{N_1}f
 \end{array}
 \right)^{int}
\end{array} \nonumber
\end{equation}
it follows that 
\begin{equation}\nonumber
Tu=Wu+\left(K'+\frac{1}{2}\right)V^{-1}\left(K+\frac{1}{2}\right)u-\left(K'+\frac{1}{2}\right)V^{-1}\tilde{N_0}f+\tilde{N_1}f
\end{equation}
or 
\begin{equation}\nonumber
Tu=Su-\left(K'+\frac{1}{2}\right)V^{-1}\tilde{N_0}f+\tilde{N_1}f,
\end{equation}
where $N_0$, and $N_1$ - Newton potentials, that corresponds to the integration over a volume.

\begin{eqnarray} \nonumber
\left.<SD,\eta>\right|_{\Gamma\setminus\Gamma_D}&=&\left.<Tu,\eta>\right|_{\Gamma\setminus\Gamma_D}
+\left.<\left(K'+\frac{1}{2}\right)V^{-1}\tilde{N_0}f,\eta>\right|_{\Gamma\setminus\Gamma_D}\nonumber \\[2ex]
&-&\left.<\tilde{N_1}f,\eta>\right|_{\Gamma\setminus\Gamma_D}-\left.<SD_0,\eta>\right|_{\Gamma\setminus\Gamma_D},
\end{eqnarray}
where $D$ - unknown displacement that lives on the non-Dirichlet part of the boundary. $\eta$ - test function that lives in the same discrete subset as $D$. $D_0$ - prescribed Dirichlet data on the boundary part $\Gamma_D$. $\Gamma$ - whole boundary. It should be mentioned that action of Poincar\'e-Steklov on the $D_0$ has integration over $\Gamma_D$ inside.

\subsubsection{Symmetric Boundary Elements with ' Poincar\'e-Steklov operator}
\begin{equation}
\begin{array}{c}
\left( \begin{array}{l}
 u \\
 Tu
 \end{array}
 \right)^{int}=\left( \begin{array}{cc}
 -K+\frac{1}{2} & V \\
 W & K'+\frac{1}{2}
 \end{array}
 \right)
 \left( \begin{array}{l}
  u \\
  Tu
 \end{array}
 \right)^{int}
 +\left( \begin{array}{l}
  \widetilde{N^a_0}\varepsilon^a \\
  \widetilde{N^a_1}\varepsilon^a
 \end{array}
 \right)^{int}

-\left( \begin{array}{l}
  0 \\
\widetilde{Tu}
 \end{array}
 \right)
\end{array} \nonumber
\end{equation}

where

\begin{equation}\nonumber
\widetilde{Tu}_i=C_{ijkl}\varepsilon^a_{kl}n_j
\end{equation}



it follows that 
\begin{equation}
Tu=Wu+\left(K'+\frac{1}{2}\right)V^{-1}\left(K+\frac{1}{2}\right)u-\left(K'+\frac{1}{2}\right)V^{-1}\widetilde{N^a_0}\varepsilon^a+\widetilde{N^a_1}\varepsilon^a - \widetilde{Tu}
\end{equation}
or 
\begin{equation}\nonumber
Tu=Su-\left(K'+\frac{1}{2}\right)V^{-1}\widetilde{N^a_0}\varepsilon^a+\widetilde{N^a_1}\varepsilon^a - \widetilde{Tu}.
\end{equation}


\subsubsection{Symmetric Boundary Elements with ' Poincar\'e-Steklov operator. Regularization}
\begin{eqnarray}
\begin{array}{c}
\left( \begin{array}{l}
 u \\
 Tu
 \end{array}
 \right)^{int}=\left( \begin{array}{cc}
 -K+\frac{1}{2} & V \\
 W & K'+\frac{1}{2}
 \end{array}
 \right)
 \left( \begin{array}{l}
  u \\
  Tu
 \end{array}
 \right)^{int}
 +\left( \begin{array}{l}
  \widetilde{N_0}\widetilde{f} \\
  \widetilde{N_1}\widetilde{f}
 \end{array}
 \right)^{int} \\[4ex]
+\left( \begin{array}{l}
  \widetilde{V}\widetilde{Tu} \\
  \widetilde{(K'+\frac{1}{2})}\widetilde{Tu}
 \end{array}
 \right)^{int}
-\left( \begin{array}{l}
  0 \\
\widetilde{Tu}
 \end{array}
 \right),
\end{array} \nonumber
\end{eqnarray}

where

\begin{equation}\label{Notation_1}
\left.
\begin{array}{rcl}
\widetilde{f}_j&=&-2G\left(\varepsilon_{ij,i}^a+\frac{\nu}{1-2\nu}e_{,j}\right),\\[4ex]
\widetilde{Tu}_i&=&2G\left(\varepsilon_{ij}^a n_j+\frac{\nu}{1-2\nu}e\,n_i\right),
\end{array}\right\}
\end{equation}
it follows that 
\begin{eqnarray}
Tu=Wu+\left(K'+\frac{1}{2}\right)V^{-1}\left(K+\frac{1}{2}\right)u&-&\left(K'+\frac{1}{2}\right)V^{-1}\widetilde{N_0}\widetilde{f}+\widetilde{N_1}\widetilde{f}- \widetilde{Tu}\nonumber \\
&-&\left(K'+\frac{1}{2}\right)V^{-1}\widetilde{V}\widetilde{Tu}+\widetilde{(K'+\frac{1}{2})}\widetilde{Tu}, \nonumber
\end{eqnarray}
or 
\begin{equation}\nonumber
Tu=Su-\left(K'+\frac{1}{2}\right)V^{-1}\widetilde{N_0}\widetilde{f}+\widetilde{N_1}\widetilde{f}-\left(K'+\frac{1}{2}\right)V^{-1}\widetilde{V}\widetilde{Tu}+\widetilde{(K'+\frac{1}{2})}\widetilde{Tu} - \widetilde{Tu}.
\end{equation}
Using the notation (\ref{Notation_1}) we can perform integration by part for  $\widetilde{N^a_0}$, $\widetilde{N^a_1}$
\begin{eqnarray}
\widetilde{N^a_0} \varepsilon^a & = & \widetilde{N_0} \widetilde{f} + \widetilde{V} \widetilde{Tu}, \nonumber \\
\widetilde{N^a_1} \varepsilon^a & = & \widetilde{N_1} \widetilde{f} + \widetilde{(K'+\frac{1}{2})}\widetilde{Tu}. \nonumber
\end{eqnarray}
Defining 
\begin{eqnarray}
\widehat{Tu}& := & \widetilde{Tu} +Tu, \nonumber \\
\widehat{f}& := & \widetilde{f} +f, \nonumber 
\end{eqnarray}
we get formulation as in Section \ref{sec:BEMPLAST}.


\subsection{Hypoelasto-viscoplasticity}\label{sec:Numeric:LargeStrain}
\subsubsection{Integral computation/implementation}

\begin{equation}\nonumber
\stackrel{*}{\tau}_{ij}=\dot{s}_{ij}+\left(\sigma_{ik}d_{jk}+\sigma_{kj}d_{ik}\right)-\sigma_{ik}v_{j,k},
\end{equation}
\begin{equation}\nonumber
\stackrel{*}{\tau}_{ij}=\dot{s}_{ji}+\stackrel{(*)}{G}_{jikl}v_{k,l},
\end{equation}

where 
\begin{equation}\nonumber
\stackrel{*}{\tau}_{ij}=\dot{s}_{ji}+\stackrel{(*)}{G}_{jikl}v_{k,l}
\end{equation}
\begin{equation}\nonumber
d_{ij}:=\dfrac{1}{2}\left(\dfrac{\partial v_i}{\partial x_j}+\dfrac{\partial v_j}{\partial x_i}\right)=:\dfrac{1}{2}(v_{i,j}+v_{j,i}),
\end{equation}

\begin{eqnarray}
\sigma_{ik}d_{jk}+\sigma_{kj}d_{ik}-\sigma_{ik}v_{j,k} &=& \dfrac{1}{2}\left(\sigma_{ik}v_{j,k}+\sigma_{ik}v_{k,j}+\sigma_{kj}v_{i,k}+\sigma_{kj}v_{k,i}\right)-\sigma_{ik}v_{j,k} \nonumber \\
&=& \dfrac{1}{2}\left(\sigma_{il}v_{k,l}\delta_{jk}+\sigma_{ik}v_{k,l}\delta_{jl}+\sigma_{lj}v_{k,l}\delta_{ki}+\sigma_{kj}v_{k,l}\delta_{li}\right)-\sigma_{il}v_{k,l}\delta_{kj}. \nonumber
\end{eqnarray}

We can write $\stackrel{(*)}{G}_{ijkl}$ :
\begin{equation}\nonumber
\stackrel{(*)}{G}_{ijkl}:=\dfrac{1}{2}\left(\sigma_{il}\delta_{jk}+\sigma_{ik}\delta_{jl}+\sigma_{lj}\delta_{ki}+\sigma_{kj}\delta_{li}\right)-\sigma_{il}\delta_{kj}.
\end{equation}

We multiply the rate of the equilibrium equation
\begin{equation}\nonumber
\dot{s}_{ji,j}+\rho_0\dot{F}^0_i=0
\end{equation}
with a test function $\tilde{v}\in \stackrel{0}{H^1}$, (ansatz function $v \in H_{DIR}+\stackrel{0}{H^1}$): 
\begin{equation}\nonumber
\int_{B^0}\left[\dot{s}_{ji,j}\tilde{v}_i+\rho_0\dot{F}^0_i\tilde{v}_i\right]=0.
\end{equation}
Integration by parts yields
\begin{equation}\nonumber
\int_{B^0}\dot{s}_{ji}\tilde{v}_{i,j}-\int_{\partial B^0}\dot{s}_{ji}n^0_{j}\tilde{v}_i-\int_{B^0}\rho_0\dot{F}^0_i\tilde{v}_i=0.
\end{equation}
Using $\dot{\tau}_i:=\dot{s}_{ji}n^0_{j}$, and $\dot{s}_{ji}=\stackrel{*}{\tau}_{ij}-\stackrel{(*)}{G}_{jikl}v_{k,l}$
we have 
\begin{equation}\nonumber
\int_{B^0}\stackrel{*}{\tau}_{ij}\tilde{v}_{i,j}-\int_{B^0}\stackrel{(*)}{G}_{jikl}v_{k,l}\tilde{v}_{i,j}-\int_{\partial B^0}\dot{\tau}_i\tilde{v}_i-\int_{B^0}\rho_0\dot{F}^0_i\tilde{v}_i=0.
\end{equation}
Using $\stackrel{*}{\tau}_{ij}=\lambda d^{(e)}_{kk}\delta_{ij}+2\mu d^{(e)}_{ij}$ and $d_{ij}:= d^{(e)}_{ij}+ d^{(n)}_{ij}$ we have
\begin{equation}\nonumber
\int_{B^0}C_{ijkl}d^{(e)}_{kl}\tilde{d}_{ij}-\int_{B^0}\stackrel{(*)}{G}_{jikl}v_{k,l}\tilde{v}_{i,j}-\int_{\partial B^0}\dot{\tau}_i\tilde{v}_i-\int_{B^0}\rho_0\dot{F}^0_i\tilde{v}_i=0.
\end{equation}
Hence,
\begin{equation}\label{eq:DiscreteEquation}
\int_{B^0}C_{ijkl}d_{kl}\tilde{d}_{ij}-\int_{B^0}\stackrel{(*)}{G}_{jikl}v_{k,l}\tilde{v}_{i,j}-\int_{\partial B^0}\dot{\tau}_i\tilde{v}_i-\int_{B^0}\rho_0\dot{F}^0_i\tilde{v}_i=-\int_{B^0}C_{ijkl}d^{(n)}_{kl}\tilde{d}_{ij},
\end{equation}
where \begin{equation}
\int_{B^0}C_{ijkl}d_{kl}\tilde{d}_{ij}
\end{equation}
is a standard FEM matrix of the Lam\'e operator in the theory of linear elasticity. The subroutine  \textbf{stiff2lame} located in \textit{libfem2.f90}  computes a contribution of one element to the  global stiffness matrix.

Since the second integral in (\ref{eq:DiscreteEquation}) is more complex we examine it it more carefully.  We represent a test function and an ansatz function on the finite element $\square$ as $\tilde{v}=\left(\begin{array}{c} \tilde{v}_1 \\ \tilde{v}_2 \end{array}\right)\tilde{\psi}(\bv{x})$,  $v=\left(\begin{array}{c} v_1 \\ v_2 \end{array}\right)\psi(\bv{x})$ respectively. Then the second integral $\int_{B^0}\stackrel{(*)}{G}_{jikl}v_{k,l}\tilde{v}_{i,j}$ in (\ref{eq:DiscreteEquation}) is a sum of local integrals having the form
\begin{eqnarray}
&&\hspace{-2cm}\int_{\square}\stackrel{(*)}{G}_{jikl}v_{k,l}\tilde{v}_{i,j}\, dx_1 dx_2=\nonumber\\
&&\int^1_{-1}\int^1_{-1} \left|\dfrac{\partial x}{\partial \xi}\right|\sum_{i=1}^{2}\sum_{j=1}^{2}\sum_{k=1}^{2}\sum_{l=1}^{2}\sum_{p=1}^{2}\sum_{q=1}^{2}\tilde{v}_i   \dfrac{\partial \tilde{\psi}}{\partial \xi_p}   \dfrac{ \partial \xi_p}{ \partial x_j}\stackrel{(*)}{G}_{jikl} \dfrac{ \partial \xi_q}{\partial  x_l}  \dfrac{\partial \psi}{\partial \xi_q}    v_k \, d\xi_1 d\xi_2.\nonumber
\end{eqnarray}
The subroutine \textbf{stiff2lameplvijCjiklukl} located in \textit{comp22pl.f90} computes a local matrix for one mesh element
\begin{equation}\nonumber
\mbox{ tmat(p,q,i,k) }:= \dfrac{ \partial \xi_p}{ \partial x_j}\stackrel{(*)}{G}_{jikl} \dfrac{ \partial \xi_q}{\partial  x_l}
\end{equation}
The subroutine  \textbf{rseiteid} located in \textit{comp2c.f90.f90}  computes a contribution of the given Neumann data to the right hand side, i.e. the integral
\begin{equation}\nonumber
 \int_{\partial B^0}\dot{\tau}_i\tilde{v}_i.
\end{equation}
The subroutine  \textbf{lcomp2} located in \textit{comp22.f90} computes a contribution of the given volume data to the right hand side, i.e. the integral
\begin{equation}\nonumber
\int_{B^0}\rho_0\dot{F}^0_i\tilde{v}_i.
\end{equation}
The subroutine  \textbf{lftgrdcomp2} located in \textit{compstiff2.f90.f90}  computes a contribution of one element to the  right hand side for the volume integral
\begin{equation}\nonumber
\int_{B^0}C_{ijkl}d^{(n)}_{kl}\tilde{d}_{ij}.
\end{equation}

